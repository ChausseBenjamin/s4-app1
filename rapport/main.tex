\documentclass[a11paper]{article}

\usepackage{karnaugh-map}
\usepackage{tabularx}
\usepackage{titlepage}
\usepackage{document}
\usepackage{booktabs}
\usepackage{multicol}
\usepackage{float}
\usepackage[usenames,dvipsnames]{xcolor}

\title{Rapport d'APP}

\class{Logique Combinatoire}
\classnb{GEN420 \& GEN430}

\teacher{Marwan Besrour \& Gabriel Bélanger}

\author{
  \addtolength{\tabcolsep}{-0.4em}
  \begin{tabular}{rcl} % Ajouter des auteurs au besoin
      Benjamin Chausse & -- & CHAB1704 \\
      Shawn Couture    & -- & COUS1912 \\
  \end{tabular}
}

\newcommand{\todo}[1]{\begin{color}{Red}\textbf{TODO:} #1\end{color}}
\newcommand{\note}[1]{\begin{color}{Orange}\textbf{NOTE:} #1\end{color}}
\newcommand{\fixme}[1]{\begin{color}{Fuchsia}\textbf{FIXME:} #1\end{color}}
\newcommand{\question}[1]{\begin{color}{ForestGreen}\textbf{QUESTION:} #1\end{color}}

\begin{document}
\maketitle

\todo{test} \fixme{another test} \note{interesting} \question{wtf}

\section{Module thermo2bin}

\subsection{Démarche}

\subsection{Implémentation}

\section{Simulation Complète}

\section{Démarche d'analyse de compatibilité}
La première étape à été d'analyser le signal d'entrée de la carte thermométrique. La DEL 2 est connecté au connecteur JD1 en
configuration "pull-up". Ceci dit, il faut donc avoir un 0 logique à son entrée pour l'allumé. Cependant, trois inverseurs sont
connecté en série avant elle. Soit deux 74V1T04STR et un NC7SP04P5X alimenté à +1.2 Volts. Après analyse de V_{OH}, V_{OL}, V_{IL}
et V_{IH} des deux portes logiques, le NC7SP04P5X à un V_{OH} de maximum 1.1 volts tandis que le 74V1T04STR à besoin d'un V_{IH}
minimum de 2 Volts. Donc selon le 74V1T04STR, le NC7SP04P5X envoie toujours un niveau logique bas, ce qui résulte à une del
tout le temps allumé sauf lorsque le connecteur de test de la del est en mode de test car un bon niveau logique haut est envoyé à
l'entrée des 74V1T04STR.


\end{document}
