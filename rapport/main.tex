\documentclass[a11paper]{article}

\usepackage{karnaugh-map}
\usepackage{tabularx}
\usepackage{titlepage}
\usepackage{document}
\usepackage{booktabs}
\usepackage{multicol}
\usepackage{float}
\usepackage{varwidth}
% \usepackage[toc,page]{appendix}
\usepackage[usenames,dvipsnames]{xcolor}

\title{Rapport d'APP}

\class{Logique Combinatoire}
\classnb{GEN420 \& GEN430}

\teacher{Marwan Besrour \& Gabriel Bélanger}

\author{
  \addtolength{\tabcolsep}{-0.4em}
  \begin{tabular}{rcl} % Ajouter des auteurs au besoin
      Benjamin Chausse & -- & CHAB1704 \\
      Shawn Couture    & -- & COUS1912 \\
  \end{tabular}
}

\newcommand{\todo}[1]{\begin{color}{Red}\textbf{TODO:} #1\end{color}}
\newcommand{\note}[1]{\begin{color}{Orange}\textbf{NOTE:} #1\end{color}}
\newcommand{\fixme}[1]{\begin{color}{Fuchsia}\textbf{FIXME:} #1\end{color}}
\newcommand{\question}[1]{\begin{color}{ForestGreen}\textbf{QUESTION:} #1\end{color}}

\begin{document}
\maketitle
\newpage
\tableofcontents
\newpage

\todo{test} \fixme{another test} \note{interesting} \question{wtf}

\section{Module thermo2bin}

\subsection{Démarche}

\subsection{Implémentation}

\section{Simulation Complète}

\section{Démarche d'analyse de compatibilité}
La première étape à été d'analyser le signal d'entrée de la carte thermométrique. La DEL 2 est connecté au connecteur JD1 en
configuration "pull-up". Ceci dit, il faut donc avoir un 0 logique à son entrée pour l'allumé. Cependant, trois inverseurs sont
connecté en série avant elle. Soit deux 74V1T04STR et un NC7SP04P5X alimenté à +1.2 Volts. Après analyse de V_{OH}, V_{OL}, V_{IL}
et V_{IH} des deux portes logiques, le NC7SP04P5X à un V_{OH} de maximum 1.1 volts tandis que le 74V1T04STR à besoin d'un V_{IH}
minimum de 2 Volts. Donc selon le 74V1T04STR, le NC7SP04P5X envoie toujours un niveau logique bas, ce qui résulte à une del
tout le temps allumé sauf lorsque le connecteur de test de la del est en mode de test car un bon niveau logique haut est envoyé à
l'entrée des 74V1T04STR.

\newpage
\appendix
\section{Code VHDL}
\todo{Finish this this}

\begin{figure}[H]
	\tiny
	\centering
	\begin{varwidth}{\linewidth}
		\begin{verbatim}
architecture Behavioral of Thermo2Bin is
  signal first_segment_of_four : STD_LOGIC_VECTOR(3 downto 0);
  signal second_segment_of_four : STD_LOGIC_VECTOR(3 downto 0);
  signal third_segment_of_four : STD_LOGIC_VECTOR(3 downto 0);

  component Add4Bits is Port (
    A : in STD_LOGIC_VECTOR (3 downto 0);
    B : in STD_LOGIC_VECTOR (3 downto 0);
    C : in STD_LOGIC;
    R : out STD_LOGIC_VECTOR (3 downto 0);
    Rc : out STD_LOGIC);
  end component;

  signal first_plus_second : STD_LOGIC_VECTOR(3 downto 0);
  signal carry_out_first_plus_second : STD_LOGIC;
  signal last_carry_out : STD_LOGIC;

begin

  first_segment_of_four(3) <= '0';
  first_segment_of_four(2) <= thermo_bus(11);
  first_segment_of_four(1) <= NOT thermo_bus(11) AND thermo_bus(9);
  first_segment_of_four(0) <= (NOT thermo_bus(11) AND thermo_bus(10)) OR (NOT thermo_bus(9) AND thermo_bus(8));

  second_segment_of_four(3) <= '0';
  second_segment_of_four(2) <= thermo_bus(7);
  second_segment_of_four(1) <= NOT thermo_bus(7) AND thermo_bus(5);
  second_segment_of_four(0) <= (NOT thermo_bus(7) AND thermo_bus(6)) OR (NOT thermo_bus(5)  AND thermo_bus(4));

  third_segment_of_four(3) <= '0';
  third_segment_of_four(2) <= thermo_bus(3);
  third_segment_of_four(1) <= NOT thermo_bus(3) AND thermo_bus(1);
  third_segment_of_four(0) <= (NOT thermo_bus(3) AND thermo_bus(2)) OR (NOT thermo_bus(1)  AND thermo_bus(0));

  first_plus_second_adder : Add4Bits port map (
    A  => first_segment_of_four,
    B  => second_segment_of_four,
    R  => first_plus_second,
    Rc => carry_out_first_plus_second,
    C  => '0');

  plus_third_adder : Add4Bits port map (
    A  => first_plus_second,
    B  => third_segment_of_four,
    R  => binary_out,
    Rc => last_carry_out,
    C  => carry_out_first_plus_second);

  error <= (
  (thermo_bus(11) AND NOT thermo_bus(10)) OR
  (thermo_bus(10) AND NOT thermo_bus(9)) OR
  (thermo_bus(9)  AND NOT thermo_bus(8)) OR
  (thermo_bus(8)  AND NOT thermo_bus(7)) OR
  (thermo_bus(7)  AND NOT thermo_bus(6)) OR
  (thermo_bus(6)  AND NOT thermo_bus(5)) OR
  (thermo_bus(5)  AND NOT thermo_bus(4)) OR
  (thermo_bus(4)  AND NOT thermo_bus(3)) OR
  (thermo_bus(3)  AND NOT thermo_bus(2)) OR
  (thermo_bus(2)  AND NOT thermo_bus(1)) OR
  (thermo_bus(1)  AND NOT thermo_bus(0)));

end Behavioral;
\end{verbatim}

	\end{varwidth}
	\caption{Module Thermo2bin}
\end{figure}

\begin{figure}[H]
	\tiny
	\centering
	\begin{varwidth}{\linewidth}
		\begin{verbatim}
architecture Behavioral of Add4Bits is

  signal bufA : STD_LOGIC;
  signal bufB : STD_LOGIC;
  signal bufC : STD_LOGIC;

  component Add1BitA is Port (
    X : in STD_LOGIC;
    Y : in STD_LOGIC;
    Ci: in STD_LOGIC;
    O : out STD_LOGIC;
    Co: out STD_LOGIC);
  end component;

  component Add1BitB is Port (
    X : in STD_LOGIC;
    Y : in STD_LOGIC;
    Ci: in STD_LOGIC;
    O : out STD_LOGIC;
    Co: out STD_LOGIC);
  end component;

begin

  first : Add1BitB port map (
    X  => A(0),
    Y  => B(0),
    Ci => C,
    O => R(0),
    Co => bufA);

  sec : Add1BitB port map (
    X  => A(1),
    Y  => B(1),
    Ci => bufA,
    O => R(1),
    Co => bufB);

  third : Add1BitA port map (
    X  => A(2),
    Y  => B(2),
    Ci => bufB,
    O => R(2),
    Co => bufC);

  fourth : Add1BitA port map (
    X  => A(3),
    Y  => B(3),
    Ci => bufC,
    O => R(3),
    Co => Rc);

end Behavioral;
\end{verbatim}

	\end{varwidth}
	\caption{Module Add4Bits}
\end{figure}

\begin{figure}[H]
	\tiny
	\centering
	\begin{varwidth}{\linewidth}
		\begin{verbatim}
architecture Behavioral of Add1BitA is

begin

  O  <= (X xor Y) xor Ci;
  Co <= ((X xor Y) and Ci) or (X and Y);

end;
\end{verbatim}

	\end{varwidth}
	\caption{Module Add1BitA}
\end{figure}

\begin{figure}[H]
	\tiny
	\centering
	\begin{varwidth}{\linewidth}
		\begin{verbatim}
architecture Behavioral of Add1BitB is

begin

Adder: process(X, Y, Ci) is variable buf: STD_LOGIC_VECTOR(2 downto 0);
begin
  buf(0) := X;
  buf(1) := Y;
  buf(2) := Ci;

  case (buf) is
    when "000" =>
      O  <= '0';
      Co <= '0';
    when "001" =>
      O  <= '1';
      Co <= '0';
    when "010" =>
      O  <= '1';
      Co <= '0';
    when "011" =>
      O  <= '0';
      Co <= '1';
    when "100" =>
      O  <= '1';
      Co <= '0';
    when "101" =>
      O  <= '0';
      Co <= '1';
    when "110" =>
      O  <= '0';
      Co <= '1';
    when "111" =>
      O  <= '1';
      Co <= '1';
    when others =>
      O  <= '0';
      Co <= '0';
  end case;

end process Adder;

end Behavioral;
\end{verbatim}

	\end{varwidth}
	\caption{Module Add1BitB}
\end{figure}

\section{Schémas Bloc}

\todo{Simulations}


\begin{figure}[H]
	\centering
	\includegraphics[width=\textwidth]{assets/img/schematic-thermo2bin.png}
	\caption{Module Thermo2bin}
\end{figure}

\begin{figure}[H]
	\centering
	\includegraphics[width=.6\textwidth]{assets/img/schematic-add4bits.png}
	\caption{Module Add4Bits}
\end{figure}

\begin{figure}[H]
	\centering
	\includegraphics[width=.6\textwidth]{assets/img/schematic-add1bita.png}
	\caption{Module Add1BitA}
\end{figure}

\begin{figure}[H]
	\centering
	\includegraphics[width=.6\textwidth]{assets/img/schematic-add1bitb.png}
	\caption{Module Add1BitB}
\end{figure}


\newpage
\section{Tables de Vérité et Karnaugh}

% \begin{figure}[H]
% \centering
% \begin{karnaugh-map}[4][4][1][$D$][$C$][$B$][$A$]
% \manualterms{
% 	0,1,2,3,
% 	4,5,6,7,
% 	8,9,10,11,
% 	12,13,14,15}
% \implicant{2}{10}
% \implicant{4}{13}
% \implicant{12}{10}
% \implicantedge{3}{2}{11}{10}
% \end{karnaugh-map}
% \caption{TEMPLATE NOT GOOD}
% \end{figure}


\begin{table}[H]
	\centering
	\caption{Table de vérité des Bits}
	\label{tab:table-de-vérité-thermométrique-4-bits}
	\vspace{.2cm}
	\begin{tabular}{llllllll}
		\toprule
		A & B & C & D & E & F & G & H \\
		\midrule
		0 & 0 & 0 & 0 & 0 & 0 & 0 & 0 \\
		0 & 0 & 0 & 1 & 0 & 0 & 0 & 1 \\
		0 & 0 & 1 & 1 & 0 & 0 & 1 & 0 \\
		0 & 1 & 1 & 1 & 0 & 0 & 1 & 1 \\
		1 & 1 & 1 & 1 & 0 & 1 & 0 & 0 \\
		\bottomrule
	\end{tabular}

\end{table}

\begin{figure}[H]
\centering
\begin{karnaugh-map}[4][4][1][$D$][$C$][$B$][$A$]
\manualterms{
	0,1,X,0,
	X,X,X,1,
	X,X,X,X,
	X,X,X,0}
\implicant{1}{9}
\implicant{4}{6}
\end{karnaugh-map}
\caption{Karnaugh pour le bit $H$}
\label{tab:karnaugh-bit-H}
\end{figure}

\begin{figure}[H]
\centering
\begin{karnaugh-map}[4][4][1][$D$][$C$][$B$][$A$]
\manualterms{
	0,0,X,1,
	X,X,X,1,
	X,X,X,X,
	X,X,X,0}
\implicant{3}{6}
\end{karnaugh-map}
\caption{Karnaugh pour le bit $G$}
\label{tab:karnaugh-bit-G}
\end{figure}

\end{document}
